\documentclass[11pt,]{article}
\usepackage{lmodern}
\usepackage{amssymb,amsmath}
\usepackage{ifxetex,ifluatex}
\usepackage{fixltx2e} % provides \textsubscript
\ifnum 0\ifxetex 1\fi\ifluatex 1\fi=0 % if pdftex
  \usepackage[T1]{fontenc}
  \usepackage[utf8]{inputenc}
\else % if luatex or xelatex
  \ifxetex
    \usepackage{mathspec}
  \else
    \usepackage{fontspec}
  \fi
  \defaultfontfeatures{Ligatures=TeX,Scale=MatchLowercase}
\fi
% use upquote if available, for straight quotes in verbatim environments
\IfFileExists{upquote.sty}{\usepackage{upquote}}{}
% use microtype if available
\IfFileExists{microtype.sty}{%
\usepackage{microtype}
\UseMicrotypeSet[protrusion]{basicmath} % disable protrusion for tt fonts
}{}
\usepackage[margin=0.95in]{geometry}
\usepackage{hyperref}
\hypersetup{unicode=true,
            pdfborder={0 0 0},
            breaklinks=true}
\urlstyle{same}  % don't use monospace font for urls
\usepackage{longtable,booktabs}
\usepackage{graphicx,grffile}
\makeatletter
\def\maxwidth{\ifdim\Gin@nat@width>\linewidth\linewidth\else\Gin@nat@width\fi}
\def\maxheight{\ifdim\Gin@nat@height>\textheight\textheight\else\Gin@nat@height\fi}
\makeatother
% Scale images if necessary, so that they will not overflow the page
% margins by default, and it is still possible to overwrite the defaults
% using explicit options in \includegraphics[width, height, ...]{}
\setkeys{Gin}{width=\maxwidth,height=\maxheight,keepaspectratio}
\IfFileExists{parskip.sty}{%
\usepackage{parskip}
}{% else
\setlength{\parindent}{0pt}
\setlength{\parskip}{6pt plus 2pt minus 1pt}
}
\setlength{\emergencystretch}{3em}  % prevent overfull lines
\providecommand{\tightlist}{%
  \setlength{\itemsep}{0pt}\setlength{\parskip}{0pt}}
\setcounter{secnumdepth}{5}
% Redefines (sub)paragraphs to behave more like sections
\ifx\paragraph\undefined\else
\let\oldparagraph\paragraph
\renewcommand{\paragraph}[1]{\oldparagraph{#1}\mbox{}}
\fi
\ifx\subparagraph\undefined\else
\let\oldsubparagraph\subparagraph
\renewcommand{\subparagraph}[1]{\oldsubparagraph{#1}\mbox{}}
\fi

%%% Use protect on footnotes to avoid problems with footnotes in titles
\let\rmarkdownfootnote\footnote%
\def\footnote{\protect\rmarkdownfootnote}

%%% Change title format to be more compact
\usepackage{titling}

% Create subtitle command for use in maketitle
\newcommand{\subtitle}[1]{
  \posttitle{
    \begin{center}\large#1\end{center}
    }
}

\setlength{\droptitle}{-2em}

  \title{}
    \pretitle{\vspace{\droptitle}}
  \posttitle{}
    \author{}
    \preauthor{}\postauthor{}
    \date{}
    \predate{}\postdate{}
  
\usepackage{hyperref}
\hypersetup{ colorlinks=true, linkcolor=blue, filecolor=cyan, urlcolor=magenta, }
\addtolength{\skip\footins}{4pc plus 10pt}
\usepackage{ragged2e}
\usepackage{xcolor}
\usepackage{fontawesome}
\usepackage{fancyhdr}
\usepackage{lipsum}
\pagestyle{fancy}
\fancyhead[LO,LE]{Universidad de Chile}
\fancyhead[RO,RE]{Tópicos en Economía y Negocios Utilizando R}
\fancyfoot[C,C]{\thepage}
\fancypagestyle{plain}{\pagestyle{fancy}}
\renewcommand{\headrulewidth}{0.4pt}% Default \headrulewidth is 0.4pt
\renewcommand{\footrulewidth}{0.4pt}% Default \footrulewidth is 0pt
\setlength{\skip\footins}{1.2pc plus 5pt minus 2pt}

\begin{document}

\begin{center}
            %\vspace{1cm}
            \includegraphics[width=1cm]{logo.jpg}\\
            \large{\textbf{ENMEC357}}\\
            \LARGE{\textbf{Tópicos en Economía y Negocios Utilizando R}}\\
            \textsc{Tarea 1}\\
            \smallskip
            \small{ \noindent \textsc{\textbf{Profesor}: \textit{Victor Macías E.}}}
            \\
            \small{ \noindent \textsc{\textbf{Ayudante}: \textit{Gabriel Cabrera G.\footnote{ \faSend: \href{mailto:gcabrerag@fen.uchile.cl}{gcabrerag@fen.uchile.cl}}}}} 
            \\
            \textsc{\textit{31 agosto 2018}}    
\end{center}

\vspace{-0.25cm}

\noindent\rule{\textwidth}{0.5pt}

\vspace{-5pt}

\begin{center}
\textbf{Instrucciones}
\end{center}

\begin{enumerate}
\def\labelenumi{\arabic{enumi}.}
\tightlist
\item
  Esta tarea debe ser entregada en grupos de máximo 3 personas.
\item
  Se debe enviar una carpeta comprimida que contenga: un archivo .Rproj
  (proyecto en R), un script y un breve informe en R Markdown con sus
  resultados, incluyendo tablas, gráficos, etc., según corresponda. El
  formato del informe puede ser word o pdf.
\item
  El asunto del email con su tarea debe ser ``Tarea 1-R 1-Apellido
  1-Apellido 2-Apellido 3''
\item
  La fecha de entrega es el Sabado 1 de Septiembre del 2018 hasta las
  23:59 hrs al correo del curso:
  \href{mailto:r2018uchile@gmail.com}{r2018uchile@gmail.com}.
\end{enumerate}

\vspace{-8pt}

\begin{center}\rule{0.5\linewidth}{\linethickness}\end{center}

\begin{quote}
``Data is a precious thing and will last longer than the systems
themselves''.

\flushright{\textit{Tim Berners-Lee}, inventor del the World Wide Web}
\end{quote}

\vspace{-10pt}

\noindent\rule{\textwidth}{0.5pt}

Una importante consultora de marketing está en la búsqueda de
estudiantes con conocimiento del lenguaje R para realizar un estudio
respecto a la industria del cine. Usted piensa que es una muy buena
oportunidad postular, considerando que está cursando un curso en el que
está aprendiendo R. Después de haber pasado varias pruebas y
entrevistas, usted es contratado y como primera tarea le entregan la
carpeta \emph{Datasets} que contiene:

\begin{longtable}[]{@{}ll@{}}
\caption{Breve descripción de las bases de datos.}\tabularnewline
\toprule
Base de datos & Descripción\tabularnewline
\midrule
\endfirsthead
\toprule
Base de datos & Descripción\tabularnewline
\midrule
\endhead
\emph{genres.csv} & Géneros de cada película.\tabularnewline
\emph{keywords.csv} & Palabras claves de cada película.\tabularnewline
\emph{production\_companies.csv} & Compañía a cargo de la
producción.\tabularnewline
\emph{production\_countries.csv} & Países donde se realizá la
producción.\tabularnewline
\emph{movie\_dataset.csv} & Base de datos principal.\tabularnewline
\bottomrule
\end{longtable}

A continuación le piden los siguientes \textbf{data frames} (sin
considerar los \emph{missing values}) pensando que en un futuro no tan
distante la información que contendrán será presentada gráficamente:

\begin{enumerate}
\def\labelenumi{\arabic{enumi}.}
\item
  Identificar las 10 palabras claves (\emph{keywords}) que más se
  repiten. (\emph{10 puntos})
\item
  Identificar:

  \begin{enumerate}
  \def\labelenumii{\alph{enumii}.}
  \tightlist
  \item
    Las 10 películas con las ingresos (\emph{revenue}) más altos.
    (\emph{10 puntos})
  \item
    Las 10 películas con las ingresos (\emph{revenue}) más bajos.
    (\emph{10 puntos})
  \item
    El número de películas con ingresos igual a cero. (\emph{10 puntos})
  \end{enumerate}
\item
  Identificar la cantidad de películas por país, luego calcular la
  utilidad\footnote{Asuma que el presupuesto fue el gasto total.}
  (\emph{revenue} - \emph{budget}) y encontrar las 5 paises con mayor y
  menor utilidad promedio. (\emph{10 puntos})
\item
  Dado que últimamente las salas IMAX han tenido una muy buena recepción
  por parte del público, la consultora le pide calcular el ingreso
  (\emph{revenue}) promedio por género, pero sólo de aquellas películas
  que fueron estrenadas en formato IMAX. (\emph{15 puntos})
\item
  La industria del cine ha cambiado mucho desde sus inicios y miles de
  películas son estrenadas cada año. Considerando sólo las películas que
  se encuentran en las bases de datos, identifique el porcentaje de
  películas estrenadas por década\footnote{Una librería útil para
    trabajar con fechas es \texttt{lubridate}. La década se puede
    construir con la ayuda del operador \texttt{\%\%}.}. (\emph{15
  puntos})
\item
  Identifique la cantidad de películas por estado de Estados Unidos (vea
  el archivo \emph{estados.csv}) y el presupuesto promedio
  (\emph{budget}) por estado. ¿Qué estados nunca fueron usados como
  lugar de filmación?, ¿Existe algún estado con presupuesto cero?.
  (\emph{15 puntos})
\end{enumerate}


\end{document}
